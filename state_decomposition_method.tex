\chapter{State-Decomposition method}

\section{State-Decomposition method}

\paragraph{}
The main topic of this project is the development of the state-decomposition method. This is suited to control problems of
environments that are composed of many sub-problems that only seldom interact with each other. Separate RL agents can be trained
for these separate sub-problems independently, thus treating them as separate problems. The separate agents are then combined at a
later stage to take into account the interactions between the sub-problems. 

\paragraph{}
The first step of the process is to identify the subsystems as groups of states. This is done by examining the state-transition
probability matrix. Given a finite MDP, this is defined as the matrix where each entry $P_{ij}$ is the probability of
transitioning from state $S_t=i$ to the next state $S_{t+1}=j$:

\begin{equation}
    \begin{bmatrix}
    P(S_{t+1}=s_0|S_{t}=s_0) & P(S_{t+1}=s_1|S_{t}=s_0) & ... & P(S_{t+1}=s_n|S_{t}=s_0) \\
    P(S_{t+1}=s_0|S_{t}=s_1) & P(S_{t+1}=s_1|S_{t}=s_1) & ... & P(S_{t+1}=s_n|S_{t}=s_1) \\
    ... & ... & ... & ... \\
    P(S_{t+1}=s_0|S_{t}=s_n) & P(S_{t+1}=s_1|S_{t}=s_n) & ... & P(S_{t+1}=s_n|S_{t}=s_n) \\
    \end{bmatrix}
\end{equation}

where $n$ is the number of states in $\mathcal{S}$. 

\begin{figure}[H]
\centering

\begin{tikzpicture}[font=\sffamily]

        % Setup the style for the states
        \tikzset{node style/.style={state, 
                                    minimum width=1cm,
                                    line width=0.5mm,
                                    fill=gray!20!white}}

        % Draw the states
        \node[node style] at (0, 0) (0) {0};
        \node[node style] at (3, 0) (1) {1};
        \node[node style] at (0, -3) (2) {2};
        \node[node style] at (3, -3) (3) {3};

        % Connect the states with arrows
        \draw[every loop,
              auto=right,
              line width=0.5mm,
              >=latex,
              draw=orange,
              fill=orange]
            (2)     edge[bend right=20]            node {} (0)
            (2)     edge[bend right=20, auto=left] node {} (1)
            (2)     edge[bend right=20, auto=left] node {} (3)
            (2)     edge[loop left=20, auto=left] node {} (2)
            (1)     edge[bend right=20]            node {} (2)
            (1)     edge[bend right=20, auto=left] node {} (0)
            (1)     edge[loop right=20, auto=left] node {} (1)
            (1)     edge[bend right=20, auto=left] node {} (3)
            (0) edge[bend right=20]            node {} (1)
            (0)     edge[bend right=20, auto=left] node {} (3)
            (0) edge[bend right=20, auto=left] node {} (2)
            (0)     edge[loop left=20, auto=left] node {} (0)
            (3) edge[bend right=20]            node {} (1)
            (3)     edge[bend right=20, auto=left] node {} (0)
            (3) edge[bend right=20, auto=left] node {} (2)
            (3)     edge[loop right=20, auto=left] node {} (3);;
\end{tikzpicture}
\begin{tikzpicture}
    \draw[->,line width=2pt]
  ([xshift=-5pt,yshift=-60pt]pic cs:starta) -- node[above,align=center] {\strut Apply \\ threshold \strut}
  ([xshift=15pt,yshift=-60pt]{{pic cs:enda}|-{pic cs:starta}});
\end{tikzpicture}
\begin{tikzpicture}[font=\sffamily]

        % Setup the style for the states
        \tikzset{node style/.style={state, 
                                    minimum width=1cm,
                                    line width=0.5mm,
                                    fill=gray!20!white}}

        % Draw the states
        \node[node style] at (0, 0) (0) {0};
        \node[node style] at (3, 0) (1) {1};
        \node[node style] at (0, -3) (2) {2};
        \node[node style] at (3, -3) (3) {3};

        % Connect the states with arrows
        \draw[every loop,
              auto=right,
              line width=0.5mm,
              >=latex,
              draw=orange,
              fill=orange]
            (2)     edge[bend right=20, auto=left] node {} (3)
            (2)     edge[loop left=20, auto=left] node {} (2)
            (1)     edge[bend right=20, auto=left] node {} (0)
            (1)     edge[loop right=20, auto=left] node {} (1)
            (0) edge[bend right=20]            node {} (1)
            (0)     edge[loop left=20, auto=left] node {} (0)
            (3) edge[bend right=20, auto=left] node {} (2)
            (3)     edge[loop right=20, auto=left] node {} (3);;
\end{tikzpicture}

\caption{Applying a threshold to the state transition matrix and normalising the probabilities can cause the MDP to separate into separate MDPs as it reduces the number of possible transitions.}

\label{fig:OneMDPtoManyMDPs}

\end{figure}

\paragraph{}
The state-decomposition method is based on the fact that some state-transition probabilities are negligible compared to others and
could be regarded as 0 to obtain a sparse version of the state-transition matrix. Specifically, every element smaller than a
threshold $\epsilon$ can be set to 0. The probabilities of the matrix can then be normalised so that each row of the matrix has a
sum of 1. This step is unnecessary and can be skipped in the implementation; however, it is necessary to obtain a valid MDP from
the state-transition matrix after applying the threshold. Following this procedure, the MDP could be formed by multiple
independent MDPs with smaller state-spaces, as shown in Figure-\ref{fig:OneMDPtoManyMDPs}. In this case, by reordering the states
it is possible to express the state-transition matrix as a block matrix in the form:

\begin{equation}
    \begin{bmatrix}       
    \fbox{$B_1$} & 0 & ... & 0  \\
     0 & \fbox{$B_2$} & ... & 0 \\
     ... & ... & ... & ... \\
     0 & 0 & ... & \fbox{$B_n$} \\
   \end{bmatrix}
\end{equation} 

where the matrices $B_1, B_2, ..., B_n$ are the state-transition matrices of the newly formed separate MDPs.

\paragraph{}
Once the states of the original MDP have been separated into multiple sub-spaces, it is possible to proceed with training the RL
agent. In the first stage of training, for each of these sub-spaces, we train a separate DQN agent. To do so, we consider for each
sub-space's agent only those state transitions that start and end within the same sub-space. Once the sub-spaces' agents
converge\footnote{Or earlier. The time when we should do this is to be researched in this project.}, these should be combined to
form a global agent that also takes into account transitions between the different sub-spaces. This is done by feeding the action
values of the sub-spaces' agents, which are the outputs of their neural networks, as inputs of another neural network, as in
Figure-\ref{fig:join_nets}. This forms a bigger neural network which is the Q-function approximator of a new global DQN agent,
which is then trained using all possible transitions. This will be referred to as the second stage of training.

\begin{figure}[H]
    \centering
    \includegraphics[width=0.5\linewidth]{figures/join_networks}
    \caption{The NNs of the decomposed states' agents are merged as input of another NN to form a new bigger NN.}
    \label{fig:join_nets}
\end{figure}

\paragraph{}
This method operates on the assumption that it would require fewer samples to achieve convergence for RL agents operating on many
smaller MDPs than on one single MDP having as the number of states the sum of the number of states of the smaller MDPs. This is
because the number of possible transitions in an MDP is the square of the number of states, so by dividing the system into
separate independent sub-problems we reduce the complexity of the control problem\footnote{By decreasing the number of possible
transitions, as \(\sum_{i=0}^n x_i^2 \leq (\sum_{i=0}^n x_i)^2\)}. By then combining the neural networks as previously explained,
we take into account interactions between the sub-problems, allowing us to achieve the optimal policy for the original MDP. For
this method to be successful, it is assumed that the weights of the neural networks of the sub-spaces' agents are not far from the
values they take once the global agent converges in the final stage of training. This is equivalent to saying that the first stage
of training provides a very good initialisation of the weights of the network used in the second stage. Although we use DQN
agents, this method could be generalised to any agents that use a function approximator to evaluate the action-values given in
input the states. This method however wouldn't be as useful if no approximator is being used, and the action-values are simply
stored in a table, such as in the original Q-learning algorithm. In this case, applying the state-decomposition would simply
ignore the transitions between different sub-spaces without changing the structure of the agent at all in the first stage of
training.

\section[Decomposing state trans. matrix]{Decomposing the state transition matrix}

\paragraph{}
The decomposition into sub-spaces of states given the state-transition table is done using Algorithm-\ref{algo:decompose}. The
Python code for this is given in the Appendix.

\begin{algorithm}[H]
\singlespacing
\SetAlgoLined

\SetKwInOut{In}{stateTransMatrix}
\SetKwInOut{Output}{subspaces}

\SetKwData{STMat}{stateTransMatrix}
\SetKwData{Out}{subspaces}

\SetKwData{Subs}{subspaces}

\SetKwData{Thre}{$\tau$}

\In{input 2D array representing the state-transition matrix}
\Output{output data structure representing the decomposed sub-spaces}

\BlankLine
\Subs = initialise as one subspace for each group;

 \For{$i \leftarrow 0$ \KwTo height(\STMat)-1}{
 \For{$j \leftarrow 0$ \KwTo width(\STMat)-1}{
  \lIf{\STMat[$i$][$j$] $>$ \Thre}{joinSubspacesContaining($i$, $j$)}
 }
 }
 
 
 \caption{State-decomposition algorithm.}
 \label{algo:decompose}
\end{algorithm}

\paragraph{}
Sometimes we know the number of sub-spaces that the original MDP should be decomposed into. For example, we might know the
characteristics of the system and the number of subsystems it can be considered to have, such as how many sub-networks compose a
certain communication network. In such case, the function that returns the decomposed sub-spaces can be run repeatedly while
updating the threshold value $\epsilon$ at each iteration using binary search, until the MDP is decomposed in the correct number
of sub-spaces. This is given in Algorithm-\ref{algo:threshold} and the Python code is in the Appendix. 

\begin{algorithm}[H]
\singlespacing
\SetAlgoLined

\SetKwInOut{In}{stateTransMatrix}
\SetKwInOut{Output}{subspaces}
\SetKwInOut{des}{nSubspacesDesired}

\SetKwData{STMat}{stateTransMatrix}
\SetKwData{Out}{subspaces}

\SetKwFunction{SD}{stateDecomposition}
\SetKwFunction{NSubs}{nSubspaces}

\SetKwData{N}{nSubspacesDesired}
\SetKwData{MaxIt}{maxIterations}
\SetKwData{Fact}{changeFactor}
\SetKwData{Thre}{$\tau$}
\SetKwData{UB}{upperBound}
\SetKwData{LB}{lowerBound}

\In{input 2D array representing the state-transition matrix}
\des{input integer representing the number of subspaces to decompose into} 
\Output{output data structure representing the decomposed subspaces}

\MaxIt $\leftarrow 50$\;
\Fact $\leftarrow 10$\;
\Thre $\leftarrow 1.0$\;
\BlankLine
\UB $\leftarrow$ NONE\;
\LB $\leftarrow$ NONE\;
\BlankLine
 \For{$i \leftarrow 1$ \KwTo \MaxIt}{
  \If{\UB and \LB are defined}{\Thre $\leftarrow (\UB + \LB) / 2$\;}
   \lIf{only \UB is defined}{\Thre $\leftarrow \UB / \Fact$}
   \lIf{only \LB is defined}{\Thre $\leftarrow \LB * \Fact$}
   \BlankLine
   \Out $\leftarrow$ \SD(\STMat, \Thre)\;
   \BlankLine
   \lIf{$\NSubs(\Out) == \N - 1$}{break}
   \lIf{$\NSubs(\Out) < \N - 1$}{\LB = \Thre}
   \lIf{$\NSubs(\Out) > \N - 1$}{\UB = \Thre}
 }
 
 
 \caption{State-decomposition into given number of subspaces}
 \label{algo:threshold}
\end{algorithm}