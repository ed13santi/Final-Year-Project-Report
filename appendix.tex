\appendix

\chapter{Performance graphs}
\label{appendix-graphs}

\begin{figure}[H]
    \centering
    \includegraphics[width=\linewidth]{figures/GridEps2DeltaSwitch.pdf}
    \caption{Performance of DQN and 'DeltaSwitch' in 'GridEps2'.}
    \label{fig:GridEps2}
\end{figure}

\begin{figure}[H]
    \centering
    \includegraphics[width=\linewidth]{figures/encodings_samples.pdf}
    \caption{Effect of different state encoding methods on the performance of the baseline DQNagent in the ’Grid’ environment.}
    \label{fig:Encodings-sample}
\end{figure}

\begin{figure}[H]
    \centering
    \includegraphics[width=\linewidth]{figures/size_comparison_samples.pdf}
    \caption{Performance of DQN with NNs of different sizes in the Taxi2 environment. ’medium2’has twice the number of trainable weights of ’medium’.}
    \label{fig:sizes-sample}
\end{figure}

\begin{figure}[H]
    \centering
    \includegraphics[width=\linewidth]{figures/GridNoWall_dqn_DeltaSwitch_samples.pdf}
    \caption{Comparison of DQN and DeltaSwitch with $\delta=0$ in the Grid environment.}
    \label{fig:gridNoWall-dqn-deltaSwitch-samples}
\end{figure}

\begin{figure}[H]
    \centering
    \includegraphics[width=\linewidth]{figures/GridNoWall_dqn_DeltaSwitch_compare_delta_samples.pdf}
    \caption{Comparison of DeltaSwitch with different $\delta$.}
    \label{fig:gridNoWall-deltaSwitch-compareDelta-samples}
\end{figure}

\begin{figure}
    \hspace{-2cm}
    \begin{tabular}{cc}
      \includegraphics[width=0.65\linewidth]{figures/geps2_0_samples.pdf} & \includegraphics[width=0.65\linewidth]{figures/geps2_001_samples.pdf} \\
      (a) $\delta=0\%$ & (b) $\delta=1\%$ \\
      \includegraphics[width=0.65\linewidth]{figures/geps2_002_samples.pdf} & \includegraphics[width=0.65\linewidth]{figures/geps2_003_samples.pdf} \\
      (c) $\delta=2\%$ & (d) $\delta=3\%$ \\
      \includegraphics[width=0.65\linewidth]{figures/geps2_004_samples.pdf} & \includegraphics[width=0.65\linewidth]{figures/geps2_005_samples.pdf} \\
      (e) $\delta=4\%$ & (f) $\delta=5\%$ \\
    \end{tabular}
    \caption{DeltaSwitch vs DQN in 'GridEps' with $0 \leq \epsilon \leq 5\%$.}
    \label{fig:deltaSwitch-GridEps-samples}
\end{figure}

\begin{figure}[H]
    \centering
    \includegraphics[width=\linewidth]{figures/effectReduced_samples.pdf}
    \caption{Effect of reducing the input size.}
    \label{fig:reduced-effect-sample}
\end{figure}

\begin{figure}[H]
    \centering
    \includegraphics[width=\linewidth]{figures/Wall-DQN-deltaSwitch_samples.pdf}
    \caption{Performance of DQN and DeltaSwitch with different decomposition in the 'Grid' environment with wall.}
    \label{fig:wall-samples}
\end{figure}

\begin{figure}[H]
    \centering
    \includegraphics[width=\linewidth]{figures/learnSwitch1_samples.pdf}
    \caption{Performance of combining 'NN0' and 'NN1' using another NN that is also fed the current state as input in a 'GridEps' environment with $\epsilon=0\%, \;1\%$.}
    \label{fig:learnSwitch1-samples}
\end{figure}

\begin{figure}[H]
    \centering
    \includegraphics[width=\linewidth]{figures/initialisedLearnSwitch_samples.pdf}
    \caption{Performance of combining 'NN0' and 'NN1' with another NN when initialising the combined network to behave as the original 'DeltaSwitch'. Tested with different network sizes}
    \label{fig:initialisedLearnSwitch-samples}
\end{figure}

\begin{figure}[H]
    \centering
    \includegraphics[width=\linewidth]{figures/Sarsa_samples.pdf}
    \caption{Comparison of using Expected SARSA after combining, progressive combined network and standard DQN.}
    \label{fig:SARSA-progressive-samples}
\end{figure}

\begin{figure}[H]
    \centering
    \includegraphics[width=\linewidth]{figures/simultaneous_samples.pdf}
    \caption{Performance of progressive network and combining network when 'NN0' and 'NN1' are not pre-trained. Standard DQN and DeltaSwitch are shown for comparison.}
    \label{fig:progressive-combining-simultaneous-samples}
\end{figure}

%\chapter{Code}
%\label{appendix-code}

%\lstset{language=Python, basicstyle=\small,breaklines=true}

%\paragraph{Function for state-decomposition}
%\begin{lstlisting}[caption={Function which is given the state-transition matrix in input and combines the thresholding of its values with the decomposition into separate MDPs, returning \texttt{d} which is a list indicating which sub-space each state belongs to (as an index) and \texttt{groups} which is a list of lists containing the states of each sub-space.}, captionpos=b]
def stateDecomposition(st_table, thr):
  def joinGroups(i, j, d, groups):
    M = max(d[i], d[j])
    m = min(d[i], d[j])
    if d[i] != d[j]:
      groups[m] += groups[M]
      del groups[M]
      for ind, el in enumerate(d):
        if el > M: d[ind] -= 1
        if el == M: d[ind] = m

  n_states = st_table.shape[0]
  d = [i for i in range(n_states)]
  groups = [[i] for i in range(n_states)]

  #iterate through all the entries
  for i, j in np.ndindex(st_table.shape):
    if st_table[i][j] > thr: joinGroups(i, j, d, groups)
          
  return d, groups
\end{lstlisting}

%\paragraph{Function for state-decomposition given target number of subspaces}
%\begin{lstlisting}[caption={Function that given the state-transition and a target number of subspaces performs the state-decomposition with different thresholds in order to obtain the correct number of subspaces.}, captionpos=b]
def stateDecompositionWithNGroups(st_table, target_n_groups):
  limitIterations = 50 # number of iterations after which the algorithm stops 
  firstChangeFactor = 10.0 # factor by which the threshold is multiplied/divided after first iteration
  thr = 1.0 # initial threshold

  lowerBound = None
  upperBound = None

  for i in range(limitIterations):
    if lowerBound == None and upperBound != None: thr = upperBound / firstChangeFactor
    if lowerBound != None and upperBound == None: thr = lowerBound * firstChangeFactor
    if lowerBound != None and upperBound != None: thr = (upperBound + lowerBound) / 2.0

    stateGroups = stateDecomposition(st_table, thr)
    n_groups = len(stateGroups[1])
    if n_groups == target_n_groups: 
      print("Target number achieved")
      break
    if n_groups < target_n_groups: lowerBound = thr
    if n_groups > target_n_groups: upperBound = thr
    print(thr)

  return stateGroups, thr
\end{lstlisting}

%\paragraph{Class for state-decomposition DQN Agent}
%\begin{lstlisting}[caption={Current version of my state-decomposition agent. In this version the states are fed in the neural networks as one-hot-encoded integers.}, captionpos=b]
class StateDecompositionDQNAgent:
  def __init__(self, state_size, action_size, subnetIndexes, subnets):
    self.state_size = state_size
    self.action_size = action_size
    self.gamma = 0.95    # discount rate
    self.epsilon = 1.0  # exploration rate
    self.epsilon_min = 0.01
    self.epsilon_decay = 0.999
    self.learning_rate = 0.001
    self.subnets = subnets
    self.n_subnets = len(stateGroups[1])
    self.subnetIndex = self._subnetIndex_init(subnetIndexes)
    self.memory = [deque(maxlen=50000) for i in range(len(self.subnets)+1)]
    self.models = self._build_models()

  def _subnetIndex_init(self, subnetIndexes):
    ind0s = subnetIndexes
    ind1s = [np.where(self.subnets[ind0s[i]] == i)[0][0] for i in range(self.state_size)]
    return list(zip(ind0s, ind1s))

  def _number_to_one_hot(self, n_classes, x):
    one_hot_vect = np.eye(n_classes)[x]
    reshaped = np.reshape(one_hot_vect, (1, n_classes))
    return reshaped

  def _one_hot_subnet(self, state):
    ind0, ind1 = self.subnetIndex[state]
    n_classes = self.subnets[ind0].shape[0]
    return self._number_to_one_hot(n_classes, ind1)

  def _all_zeros(self, subnet):
    l = len(subnet)
    return np.zeros((1,l))
    
  def _one_hot_joined(self, state):
    ind0, _ = self.subnetIndex[state]
    one_hot_inputs = []
    for i, subnet in enumerate(self.subnets):
      if i == ind0:
        one_hot_inputs.append(self._one_hot_subnet(state))
      else:
        one_hot_inputs.append(self._all_zeros(subnet))
    return np.concatenate(one_hot_inputs, axis=1)
    
  def _build_subnet_model(self, subnet):
    subLen = len(subnet)
    inputs = Input(shape=(subLen,), name="input{}".format(subLen))
    x = Dense(50, activation='relu', name="subnet{}_dense_0".format(subLen))(inputs)
    x = Dense(50, activation='relu', name="subnet{}_dense_1".format(subLen))(x)
    x = Dense(50, activation='relu', name="subnet{}_dense_2".format(subLen))(x)
    out = Dense(self.action_size, activation='linear', name="subnet{}_dense_3".format(subLen))(x)
    model = Model(inputs=inputs, outputs=out, name="model{}".format(subLen))
    model.compile(loss='mse',
                  optimizer=Adam(lr=self.learning_rate))
    return model
    
  def _build_models(self):
    return [self._build_subnet_model(s) for s in self.subnets]

  def join_models(self):
    def _joined_subnets(input):
      ind = 0
      outputs = []
      for i, subnet in enumerate(self.subnets):
        input_subnet = Lambda(lambda x: x[:,i:i+len(subnet)], name="lambda_{}".format(i))(input)
        return Lambda(lambda x: x[:,i:i+len(subnet)], name="lambda_{}".format(i))(input)
        outputs.append(self.models[i](input_subnet))
        i += len(subnet)
      return Concatenate(name="concatenate_int_outputs")(outputs)

    def _top_network(input):
      n_inputs = self.action_size * self.n_subnets
      x = Dense(24, activation='relu', input_shape=(None,n_inputs,), name="top_dense_0")(input)
      x = Dense(48, activation='relu', name="top_dense_1")(x)
      x = Dense(48, activation='relu', name="top_dense_2")(x)
      x = Dense(self.action_size, activation='linear', name="top_dense_3")(x)
      return x

    input = Input(shape=(self.state_size,), name="intermediate_input")
    int_outputs = _joined_subnets(input)
    output = _top_network(int_outputs)
    model = Model(inputs=input, outputs=output, name="joined_model")
    model.compile(loss='mse',
                  optimizer=Adam(lr=self.learning_rate)) #maybe should have separate lr for the top model
    self.models.append(model)

  def remember_subnet(self, state, action, reward, next_state, done):
    ind0, _ = self.subnetIndex[state]
    next_ind0, _ = self.subnetIndex[next_state]
    if ind0 == next_ind0:
      self.memory[ind0].append((state, action, reward, next_state, done))
      return ind0
    self.memory[self.n_subnets].append((state, action, reward, next_state, done))
    return -1

  def remember_joined(self, state, action, reward, next_state, done):
    self.memory[self.n_subnets].append((state, action, reward, next_state, done))

  def act_subnet(self, state):# We implement the epsilon-greedy policy
    if np.random.rand() > self.epsilon:
      one_hot_state = self._one_hot_subnet(state)
      model = self.models[self.subnetIndex[state][0]]
      act_values = model.predict(one_hot_state)
      return np.argmax(act_values[0]) # returns action
    return random.randrange(self.action_size)
    
  def exploit(self, state): # When we test the agent we dont want it to explore anymore, but to exploit what it has learnt
    act_values = self.models[self.n_subnets].predict(self._one_hot_joined(state))
    return np.argmax(act_values[0]) 

  def act_joined(self, state):# We implement the epsilon-greedy policy
    if np.random.rand() > self.epsilon:
      return self.exploit(state) # returns action
    return random.randrange(self.action_size)


  def replay(self, subnet_ind, batch_size, joined):
    if joined:
      index = self.n_subnets
    else: 
      index = subnet_ind
    
    minibatch = random.sample(self.memory[index], batch_size)

    b = False
    for el in minibatch:
      if self.subnetIndex[el[0]][0] != subnet_ind or self.subnetIndex[el[3]][0] != subnet_ind:
        return True
    if b: print("WRONG")

    action_b = np.squeeze(np.array(list(map(lambda x: x[1], minibatch))))
    reward_b = np.squeeze(np.array(list(map(lambda x: x[2], minibatch))))
    done_b = np.squeeze(np.array(list(map(lambda x: x[4], minibatch))))

    ### Q-learning
    if joined:
      state_b_one_hot = np.squeeze(np.array(list(map(lambda x: self._one_hot_joined(x[0]), minibatch))))
      next_state_b_one_hot = np.squeeze(np.array(list(map(lambda x: self._one_hot_joined(x[3]), minibatch))))
    else: 
      state_b_one_hot = np.squeeze(np.array(list(map(lambda x: self._one_hot_subnet(x[0]), minibatch))))
      next_state_b_one_hot = np.squeeze(np.array(list(map(lambda x: self._one_hot_subnet(x[3]), minibatch))))

    pred = self.models[index].predict(next_state_b_one_hot)
    target = (reward_b + self.gamma * np.amax(pred, 1))
    target[done_b==1] = reward_b[done_b==1]
    target_f = self.models[index].predict(state_b_one_hot)

    for k in range(target_f.shape[0]):
      target_f[k][action_b[k]] = target[k]
    self.models[index].train_on_batch(state_b_one_hot, target_f)
    if self.epsilon > self.epsilon_min:
        self.epsilon *= self.epsilon_decay

  def load(self, name):
    self.model.load_weights(name)
  def save(self, name):
    self.model.save_weights(name)
\end{lstlisting}