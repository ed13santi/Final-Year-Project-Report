\chapter{Conclusions and Further Work}

\section{Findings}

- actually I think that what makes a difference is more like a better decoding

-DeltaSwitch is basically a hierarchical agent that works for environemnts that are not perfectly decomposable too. Explain that
decomposition has basically changed and it's not reallly decomposition anymore if we are using the DeltaSwitch.

- while in the origianl it was more like solve different problems, now it is more like HAVING BASICALLY DIFFERNT PARTS OF THE BRAIN DOING DIFFERENT THINGS

\section{Discussion}

\paragraph{}
One limitation of this project is that due to timing and computational constraints, the state-decomposition technique was only
tested in relatively simple environments with small state-spaces. As this technique aims to aid in scenarios with very large
state-spaces, it would be useful to test it in such environments. However, I predict that the performance gain of
state-decomposition could increase further for very large state-spaces. The reason is that the environments used in this project
were small enough so that all states were likely to be visited multiple times and the neural networks approximating Q-values would
not have to massively rely on generalisation. However, the generalisation properties of neural networks are increasingly important
as the size of the state-space increases, as it becomes harder to visit all the states during training. As the state-decomposition
methods train a sub-agent for each sub-space, these agents only have to generalise over the smaller sub-spaces. It might be that
this effect outweights the fact that there are fewer samples for each sub-space than for the whole state-space, leading to better
generalisation and thus faster convergence.

\paragraph{}
Another problem is that while it was shown that the state-decomposition method can lead to performance gains in certain
environments, using certian network architectures and and hyperparameters, this doesn't necessarily imply that this would also
occur with different design choices. For example, all experiments shown in this report used an $\epsilon$-greedy policy and
$\gamma=0.95$, however, different values of these might affect the results.

\paragraph{}
- talk about number of trials

\paragraph{Finding the state-transition matrix}
The first step in the state-decomposition method is to obtain the state-transition matrix. Since we are testing the algorithm in
software simulations, the state-transition matrix can be easily obtained from the source code of the environments and it can be
considered as given. This is usually not the case, so at a later stage we shall test methods to approximate the state-transition
matrix. This can be done by interacting with the environment and sampling the $\langle S,A,R,S' \rangle$ tuples that are
generated. Since the state-transition probabilities are dependent on the action, the state-transition matrix is dependent on the
chosen policy. As we are conducting this before the training and we have no information about the optimal policy, we choose to use
the most general possible policy to obtain the state-transition table, that is, the actions are all chosen with equal probability
independently of the current state. The number of iterations required to construct the state-transition matrix is dependant on the
number of states, actions, the probability of visiting each different state and the required level of accuracy of the
state-transition matrix estimate. 


\section{Extensions}

\paragraph{Compression and embeddings of states} The state-decomposition algorithm divides the original state-space of the system
into smaller groups of states. It can be supposed that the variance of the states within these sub-spaces is smaller than in the
original state space. Thus, the states that are input in the sub-spaces' agents could be successfully compressed into a more
compact representation which could positively affect the performance of the algorithm. More generally, it could be beneficial to
learn custom embeddings for the states of each sub-space, having the aim of better generalisation rather than state-space
reduction.

\paragraph{Applying state-decomposition to continuous state-spaces} The state-decomposition method is only applicable to discrete
state spaces. A possible extension would investigate methods to decompose continuous state-spaces. A state-transition matrix could
be obtained by discretising the states and the state-space would be decomposed accordingly, while the agents could still be fed
the original continuous states as input.

\label{sec:different-decompositions}
\paragraph{Different types of decomposition} - Need to explain that this project focussed on decomposition using tranisiotn probabilities, however this is not the only kind
and actually the example given in the data centre part might give an even better decomposition.
\begin{itemize}
    \item \textbf{Applying threshold to transitions between whole sub-spaces} The original state-decomposition method in this project
    consists of temporarily ignoring improbable state transitions by applying a threshold to the state-transition matrix. There are
    scenarios in which groups of states are connected by state-transitions that have a high probability but it would still be useful
    to consider them as separate sub-spaces. Consider for example Figure-\ref{fig:connected-states}, where an MDP is composed of
    nearly independent sub-spaces that are only connected by the blue transitions. In such a scenario it could be useful to determine
    the sub-spaces by limiting the number of possible non-zero transition probabilities between any two sub-spaces rather than
    applying a probability threshold to every single transition. Alternatively, we could apply a threshold to transition probabilities
    between sub-spaces instead of single states.
\end{itemize}

\paragraph{Offline training} The samples obtained from the pre-training interaction required to obtain the state-transition matrix
could be saved and later added to the experience replay memories of the DQN agents. Offline pre-training could also be conducted
using these samples as this could improve sample efficiency. Similarly, transition samples between different sub-spaces are not
used in the first stage of training. These could be used to conduct offline pre-training of the merged network.

\paragraph{Sharing information between sub-agents} Consider a system composed of many idenatical sub-systems. The proposed
decomposition method would be able to exploit the structure of the system to learn more efficiently. However, in this case, it
would be possible to also exploit the similarity of the sub-systems in order to be even more efficient. This is because in this
case, the sub-agents would basically be identical, meaning that the samples belonging to one sub-space could also be used to train
the agents of the other sub-spaces. This is an extreme scenario, however, there are many common environemnts in which the
sub-systems are similar if not identical. Studying ways to share learned information between sub-agents to exploit the similarity
of the sub-spaces could further increasing the sample efficiency of training.

\begin{figure}[H]
    \centering
    \resizebox{0.5\linewidth}{!}{
    \begin{tikzpicture}[]

        % Setup the style for the states
        \tikzset{node style/.style={state, 
                                    minimum width=1cm,
                                    line width=0.5mm,
                                    fill=gray!20!white}}

        % Draw the states
        \node[node style] at (0, 0) (0) {0};
        \node[node style] at (3, 0) (1) {1};
        \node[node style] at (0, -3) (2) {2};
        \node[node style] at (3, -3) (3) {3};
        
        \node[node style] at (6, 0) (4) {4};
        \node[node style] at (9, 0) (5) {5};
        \node[node style] at (6, -3) (6) {6};
        \node[node style] at (9, -3) (7) {7};
        
        \node[node style] at (0, -6) (8) {8};
        \node[node style] at (3, -6) (9) {9};
        \node[node style] at (0, -9) (10) {10};
        \node[node style] at (3, -9) (11) {11};
        
        \node[node style] at (6, -6) (12) {12};
        \node[node style] at (9, -6) (13) {13};
        \node[node style] at (6, -9) (14) {14};
        \node[node style] at (9, -9) (15) {15};

        % Connect the states with arrows
        \draw[every loop,
                auto=right,
                line width=0.5mm,
                >=latex,
                draw=orange,
                fill=orange]
            (2)     edge[bend right=20]            node {} (0)
            (2)     edge[bend right=20, auto=left] node {} (1)
            (2)     edge[bend right=20, auto=left] node {} (3)
            (2)     edge[loop left=20, auto=left] node {} (2)
            (1)     edge[bend right=20]            node {} (2)
            (1)     edge[bend right=20, auto=left] node {} (0)
            (1)     edge[loop right=20, auto=left] node {} (1)
            (1)     edge[bend right=20, auto=left] node {} (3)
            (0) edge[bend right=20]            node {} (1)
            (0)     edge[bend right=20, auto=left] node {} (3)
            (0) edge[bend right=20, auto=left] node {} (2)
            (0)     edge[loop left=20, auto=left] node {} (0)
            (3) edge[bend right=20]            node {} (1)
            (3)     edge[bend right=20, auto=left] node {} (0)
            (3) edge[bend right=20, auto=left] node {} (2)
            (3)     edge[loop right=20, auto=left] node {} (3)
            
            (6)     edge[bend right=20]            node {} (4)
            (6)     edge[bend right=20, auto=left] node {} (5)
            (6)     edge[bend right=20, auto=left] node {} (7)
            (6)     edge[loop left=20, auto=left] node {}  (6)
            (5)     edge[bend right=20]            node {} (6)
            (5)     edge[bend right=20, auto=left] node {} (4)
            (5)     edge[loop right=20, auto=left] node {} (5)
            (5)     edge[bend right=20, auto=left] node {} (7)
            (4) edge[bend right=20]            node {}     (5)
            (4)     edge[bend right=20, auto=left] node {} (7)
            (4) edge[bend right=20, auto=left] node {}     (6)
            (4)     edge[loop left=20, auto=left] node {}  (4)
            (7) edge[bend right=20]            node {}     (5)
            (7)     edge[bend right=20, auto=left] node {} (4)
            (7) edge[bend right=20, auto=left] node {}     (6)
            (7)     edge[loop right=20, auto=left] node {} (7)
            
            (10)     edge[bend right=20]            node {} (8)
            (10)     edge[bend right=20, auto=left] node {} (9)
            (10)     edge[bend right=20, auto=left] node {} (11)
            (10)     edge[loop left=20, auto=left] node {} (10)
            (9)     edge[bend right=20]            node {} (10)
            (9)     edge[bend right=20, auto=left] node {} (8)
            (9)     edge[loop right=20, auto=left] node {} (9)
            (9)     edge[bend right=20, auto=left] node {} (11)
            (8) edge[bend right=20]            node {}     (9)
            (8)     edge[bend right=20, auto=left] node {} (11)
            (8) edge[bend right=20, auto=left] node {}     (10)
            (8)     edge[loop left=20, auto=left] node {}  (8)
            (11) edge[bend right=20]            node {}     (9)
            (11)     edge[bend right=20, auto=left] node {} (8)
            (11) edge[bend right=20, auto=left] node {}     (10)
            (11)     edge[loop right=20, auto=left] node {} (11)
            
            (14)     edge[bend right=20]            node {} (12)
            (14)     edge[bend right=20, auto=left] node {} (13)
            (14)     edge[bend right=20, auto=left] node {} (15)
            (14)     edge[loop left=20, auto=left] node {}  (14)
            (13)     edge[bend right=20]            node {} (14)
            (13)     edge[bend right=20, auto=left] node {} (12)
            (13)     edge[loop right=20, auto=left] node {} (13)
            (13)     edge[bend right=20, auto=left] node {} (15)
            (12) edge[bend right=20]            node {}     (13)
            (12)     edge[bend right=20, auto=left] node {} (15)
            (12) edge[bend right=20, auto=left] node {}     (14)
            (12)     edge[loop left=20, auto=left] node {}  (12)
            (15) edge[bend right=20]            node {}     (13)
            (15)     edge[bend right=20, auto=left] node {} (12)
            (15) edge[bend right=20, auto=left] node {}     (14)
            (15)     edge[loop right=20, auto=left] node {} (15)
            (3) edge[bend right=20, draw=blue, fill=blue]            node {}     (6)
            (6)     edge[bend right=20, auto=left, draw=blue, fill=blue] node {} (12)
            (12) edge[bend right=20, auto=left, draw=blue, fill=blue] node {}     (9)
            (9)     edge[bend right=20, auto=left, draw=blue, fill=blue] node {} (3);
    \end{tikzpicture}}
    \caption{MDP composed of smaller subspaces of states connected by few but probable transitions.}
    \label{fig:connected-states}
\end{figure}


\paragraph{Using state-decomposition for automatic sub-goal discovery} The idea of state-decomposition, especially when
implemented by considering transitions between whole sub-spaces could be applied to the automatic discovery of sub-goals in
hierarchical reinforcement learning. A similar idea has been adopted by \cite{simsek_identifying_2005} and
\cite{menache_q-cutdynamic_2002}, in which the graph of the MDP is divided into sub-spaces by cutting through the least possible
number of state-transitions. The intuition is that sub-goals, such as driving towards the passenger in "Taxi-v3" or driving to the
destination after picking it up, are often connected by bottleneck states, such as states 3, 6, 9 and 12 in
Figure-\ref{fig:connected-states}.

\paragraph{Proving mathematical bounds} Ok cool it works, but why? Mathemtical foundation is important m8

\paragraph{Applying state-decomposition to policy gradient algorithms} If we do Q-learning, why not other algos too, what about
RNNs and stuff