\chapter{Conclusions and further work}

\section{Findings}

\paragraph{}
The performance of different state-decomposition architectures was tested in various environments. It was found that the initially
proposed architecture which is trained in two stages fails to trasfer knowledge from the first to the second stage. The most
positive results were obtained with the 'DeltaSwitch' architecture with $\delta=0$, which simply assigns a different neural
network to each of the decomposed sub-spaces, but allows to train each neural network using target values generated by the other
neural networks. This obtains a performance gain over DQN in various environments, as it experiences higher rewards during
training, though not necessarily reaching convergence faster. It was identified that the most likely way in which 'DeltaSwitch'
improves performance over DQN is by simplifying the functions that the neural networks aim to learn.

\paragraph{}
Two transfer learning techniques were also tested, in order to gain insight on what could be possible techniques to use in
two-stage state-decomposition architectures. One of them uses a neural network to combine the networks of the sub-spaces, while
the other combines them using the idea of a progressive neural network. The pretrained networks were successfully used to provide
a good initialisation in the training environment, however, the rate of convergence of these techniques was the same as for DQN,
meaning that "forgetting" occurs and that knowledge is not successfully transferred from the pretrained networks to the final policy. It
was attempted to apply these techniques to a two-stage decomposition approach, however this proved to be difficult as a good
initialisation of the second stage required a long training time in the first stage.

\paragraph{}
After testing 'DeltaSwitch' in different environments it was also found that different decomposition types perform better in
different scenarios. For example, in 'GridEps2', the best performance was achieved by decomposing by position, which is different
from the initially proposed type of decomposition that relies applying a threshold to transition probabilities. This suggests that
the decomposition doesn't necessarily have to separate semi-independent problems to provide a performance gain, as long as it
simplifies the funcitons that the neural networks have to learn.

\paragraph{}
The two transfer learning techniques inspired two other techniques having only one-stage. These are basically the transfer
learning techniques in which the pretrained networks are not pretrained, but are trained simultaneously with the combining network.
The architecture using a simple combining network did not present any performance gain over DQN, while the one using a progressive
network architecture showed performance gains in some environments.

\paragraph{}
It was thus found that the state-decomposition technique can lead to "cheaper" training, meaning that the rewards are higher
during the training process. The 'DeltaSwitch' agent showed the best performance in my tests, while the progressive network
architecture showed slightly lower performance but it can be used with a wider range of decompositions. A possible application of
these techniques is in the allocation of tasks to interconnected data-centres. A performance gain was only achieved using one
stage approaches, however, two-stage approaches could be successful if a way to successfully transfer knowledge to the second
stage is determined.

\section{Discussion}

\paragraph{}
One limitation of this project is that due to timing and computational constraints, the state-decomposition technique was only
tested in relatively simple environments with relatively small state-spaces. As this technique aims to aid in scenarios with very
large state-spaces, it would be useful to test it in such environments. However, I predict that the performance gain of
state-decomposition could increase further for very large state-spaces. The reason is that the environments used in this project
were small enough so that all states were likely to be visited multiple times and the neural networks approximating Q-values would
not have to rely on generalisation. However, the generalisation properties of neural networks are increasingly important
as the size of the state-space increases, as it becomes harder to visit all the states during training. As the state-decomposition
methods train a sub-agent for each sub-space, these agents only have to generalise over the smaller sub-spaces. It might be that
this effect outweighs the fact that there are fewer samples for each sub-space than for the whole state-space, leading to better
generalisation and thus faster convergence and even better performance improvement. 

\paragraph{}
Additionally, for the tested scenarios a higher performance was achieved using bigger neural networks, due to the small size of
the state-spaces which leads to better learning with massive neural networks that simply memorize values for each possible state,
thus overfitting. However, in scenarios with big state-spaces that rely on generalisation, overfitting is very harmful. The
state-decomposition technique, by simplifying the functions that the neural networks have to learn, could allow the use of smaller
neural networks that are less likely to overfit, leading to better performance in environment with very large state-spaces.

\paragraph{}
A limitation of the state-decomposition method, at least in the explored versions, is that it fails to exploit similarities
between sub-spaces. This is because while dividing the state-space into smaller sub-spaces, it also reduces the number of
available training samples for each sub-space, as each sample is only used by one sub-space. However, some of these samples could
provide information for many similar sub-spaces. For example, in the 'Grid' environment, learning to navigate to one destination
clearly could have some carry-over to learning how to navigate to the other. A more efficient technique would be able to identify
these similarities and further improve sample-efficiency.

\paragraph{}
Another problem is that while it was shown that the state-decomposition method can lead to performance gains in certain
environments, using certain network architectures and hyperparameters, this doesn't necessarily imply that this would also
occur with different design choices. For example, all experiments shown in this report used an $\epsilon$-greedy policy and
$\gamma=0.95$, however, different values of these might affect the results.

\paragraph{}
A limitation of the experimental technique is that some of the results were obtained using a relatively small number of samples
(ranging from 20 to 100), which was not always consistent due to timing and computational power constraints. However, the results
are considered statistically relevant and it is unlikely that running further trials would have affected to conclusions of this
report.

\section{Extensions}

\paragraph{Finding the state-transition matrix}
The first step in the originally proposed state-decomposition method is to obtain the state-transition matrix, as this is usually
unknown. A possible extension would be to investigate ways to efficiently estimate this for the purposes of state-decomposition.
This could be done by interacting with the environment and sampling the $\langle S,A,R,S' \rangle$ tuples that are generated. Since
the state-transition probabilities are dependent on the action, the state-transition matrix is dependent on the chosen policy. As
this is conducted before the training, when we have no information about the optimal policy, we could choose to use the most general
possible policy to obtain the state-transition table, that is, the actions are all chosen with equal probability independently of
the current state. The number of iterations required to construct the state-transition matrix is dependant on the number of
states, actions, the probability of visiting each different state and the required level of accuracy of the state-transition
matrix estimate, which also would need to be determined. 

\paragraph{Compression and embeddings of states} The state-decomposition algorithm divides the original state-space of the system
into smaller groups of states. It can be supposed that the variance of the states within these sub-spaces is smaller than in the
original state space. Thus, the states that are input in the sub-spaces' agents could be successfully compressed into a more
compact representation which could positively affect the performance of the algorithm. More generally, it could be beneficial to
learn custom embeddings for the states of each sub-space, having the aim of better generalisation rather than state-space
reduction.

\paragraph{Applying state-decomposition to continuous state-spaces} The state-decomposition method is only applicable to discrete
state spaces. A possible extension would investigate methods to decompose continuous state-spaces. A state-transition matrix could
be obtained by discretising the states and the state-space would be decomposed accordingly, while the agents could still be fed
the original continuous states as inputs.

\label{sec:different-decompositions}
\paragraph{Different types of decomposition} This project mostly focused on decomposition using transition probabilities, however
this is not the only possible kind of decomposition, as seen in the data centres example. Example of different decompositions are:
\begin{itemize}
    \item \textbf{Applying threshold to transitions between whole sub-spaces} The original state-decomposition method consists of
    temporarily ignoring improbable state transitions by applying a threshold to the state-transition matrix. There are scenarios
    in which groups of states are connected by state-transitions that have a high probability but it would still be useful to
    consider them as separate sub-spaces. Consider for example Figure-\ref{fig:connected-states}, where an MDP is composed of
    nearly independent sub-spaces that are only connected by the blue transitions. In such a scenario it could be useful to
    determine the sub-spaces by limiting the number of possible non-zero transition probabilities between any two sub-spaces
    rather than applying a probability threshold to every single transition. Alternatively, we could apply a threshold to
    transition probabilities between whole sub-spaces instead of single states. 
    \item \textbf{Decomposing along the length of the state vector} This could be useful in a case like the example with
    data-centres, in which different transitions are only likely to affect different sections of the input state's representation
    vector. Thus, different neural networks could deal with each of these sections of the vector separately.
\end{itemize}

\paragraph{Sharing information between sub-agents} Consider a system composed of many identical sub-systems. The proposed
decomposition method would be able to exploit the structure of the system to learn more efficiently. However, in this case, it
would be possible to also exploit the similarity of the sub-systems in order to be even more efficient. This is because in this
case, the sub-agents would basically be identical, meaning that the samples belonging to one sub-space could also be used to train
the agents of the other sub-spaces. This is an extreme scenario, however, there are many environments in which the sub-systems are
similar if not identical. Studying ways to share learned information between sub-agents to exploit the similarity of the
sub-spaces could further increase the sample efficiency of training.

\begin{figure}[H]
    \centering
    \resizebox{0.5\linewidth}{!}{
    \begin{tikzpicture}[]

        % Setup the style for the states
        \tikzset{node style/.style={state, 
                                    minimum width=1cm,
                                    line width=0.5mm,
                                    fill=gray!20!white}}

        % Draw the states
        \node[node style] at (0, 0) (0) {0};
        \node[node style] at (3, 0) (1) {1};
        \node[node style] at (0, -3) (2) {2};
        \node[node style] at (3, -3) (3) {3};
        
        \node[node style] at (6, 0) (4) {4};
        \node[node style] at (9, 0) (5) {5};
        \node[node style] at (6, -3) (6) {6};
        \node[node style] at (9, -3) (7) {7};
        
        \node[node style] at (0, -6) (8) {8};
        \node[node style] at (3, -6) (9) {9};
        \node[node style] at (0, -9) (10) {10};
        \node[node style] at (3, -9) (11) {11};
        
        \node[node style] at (6, -6) (12) {12};
        \node[node style] at (9, -6) (13) {13};
        \node[node style] at (6, -9) (14) {14};
        \node[node style] at (9, -9) (15) {15};

        % Connect the states with arrows
        \draw[every loop,
                auto=right,
                line width=0.5mm,
                >=latex,
                draw=orange,
                fill=orange]
            (2)     edge[bend right=20]            node {} (0)
            (2)     edge[bend right=20, auto=left] node {} (1)
            (2)     edge[bend right=20, auto=left] node {} (3)
            (2)     edge[loop left=20, auto=left] node {} (2)
            (1)     edge[bend right=20]            node {} (2)
            (1)     edge[bend right=20, auto=left] node {} (0)
            (1)     edge[loop right=20, auto=left] node {} (1)
            (1)     edge[bend right=20, auto=left] node {} (3)
            (0) edge[bend right=20]            node {} (1)
            (0)     edge[bend right=20, auto=left] node {} (3)
            (0) edge[bend right=20, auto=left] node {} (2)
            (0)     edge[loop left=20, auto=left] node {} (0)
            (3) edge[bend right=20]            node {} (1)
            (3)     edge[bend right=20, auto=left] node {} (0)
            (3) edge[bend right=20, auto=left] node {} (2)
            (3)     edge[loop right=20, auto=left] node {} (3)
            
            (6)     edge[bend right=20]            node {} (4)
            (6)     edge[bend right=20, auto=left] node {} (5)
            (6)     edge[bend right=20, auto=left] node {} (7)
            (6)     edge[loop left=20, auto=left] node {}  (6)
            (5)     edge[bend right=20]            node {} (6)
            (5)     edge[bend right=20, auto=left] node {} (4)
            (5)     edge[loop right=20, auto=left] node {} (5)
            (5)     edge[bend right=20, auto=left] node {} (7)
            (4) edge[bend right=20]            node {}     (5)
            (4)     edge[bend right=20, auto=left] node {} (7)
            (4) edge[bend right=20, auto=left] node {}     (6)
            (4)     edge[loop left=20, auto=left] node {}  (4)
            (7) edge[bend right=20]            node {}     (5)
            (7)     edge[bend right=20, auto=left] node {} (4)
            (7) edge[bend right=20, auto=left] node {}     (6)
            (7)     edge[loop right=20, auto=left] node {} (7)
            
            (10)     edge[bend right=20]            node {} (8)
            (10)     edge[bend right=20, auto=left] node {} (9)
            (10)     edge[bend right=20, auto=left] node {} (11)
            (10)     edge[loop left=20, auto=left] node {} (10)
            (9)     edge[bend right=20]            node {} (10)
            (9)     edge[bend right=20, auto=left] node {} (8)
            (9)     edge[loop right=20, auto=left] node {} (9)
            (9)     edge[bend right=20, auto=left] node {} (11)
            (8) edge[bend right=20]            node {}     (9)
            (8)     edge[bend right=20, auto=left] node {} (11)
            (8) edge[bend right=20, auto=left] node {}     (10)
            (8)     edge[loop left=20, auto=left] node {}  (8)
            (11) edge[bend right=20]            node {}     (9)
            (11)     edge[bend right=20, auto=left] node {} (8)
            (11) edge[bend right=20, auto=left] node {}     (10)
            (11)     edge[loop right=20, auto=left] node {} (11)
            
            (14)     edge[bend right=20]            node {} (12)
            (14)     edge[bend right=20, auto=left] node {} (13)
            (14)     edge[bend right=20, auto=left] node {} (15)
            (14)     edge[loop left=20, auto=left] node {}  (14)
            (13)     edge[bend right=20]            node {} (14)
            (13)     edge[bend right=20, auto=left] node {} (12)
            (13)     edge[loop right=20, auto=left] node {} (13)
            (13)     edge[bend right=20, auto=left] node {} (15)
            (12) edge[bend right=20]            node {}     (13)
            (12)     edge[bend right=20, auto=left] node {} (15)
            (12) edge[bend right=20, auto=left] node {}     (14)
            (12)     edge[loop left=20, auto=left] node {}  (12)
            (15) edge[bend right=20]            node {}     (13)
            (15)     edge[bend right=20, auto=left] node {} (12)
            (15) edge[bend right=20, auto=left] node {}     (14)
            (15)     edge[loop right=20, auto=left] node {} (15)
            (3) edge[bend right=20, draw=blue, fill=blue]            node {}     (6)
            (6)     edge[bend right=20, auto=left, draw=blue, fill=blue] node {} (12)
            (12) edge[bend right=20, auto=left, draw=blue, fill=blue] node {}     (9)
            (9)     edge[bend right=20, auto=left, draw=blue, fill=blue] node {} (3);
    \end{tikzpicture}}
    \caption{MDP composed of smaller subspaces of states connected by few but probable transitions.}
    \label{fig:connected-states}
\end{figure}


\paragraph{Using state-decomposition for automatic sub-goal discovery} The idea of state-decomposition, especially when
implemented by considering transitions between whole sub-spaces could be applied to the automatic discovery of sub-goals in
hierarchical reinforcement learning. A similar idea has been adopted by \cite{simsek_identifying_2005} and
\cite{menache_q-cutdynamic_2002}, in which the graph of the MDP is divided into sub-spaces by cutting through the least possible
number of state-transitions. The intuition is that sub-goals, such as driving towards the passenger in "Taxi-v3" or driving to the
destination after picking it up, are often connected by bottleneck states, such as states 3, 6, 9 and 12 in
Figure-\ref{fig:connected-states}.

\paragraph{Proving mathematical bounds} It could be useful to derive mathematical bounds for the rate of convergence of different
state-decomposition algorithm. This could provide theoretical guidance for more efficient experimentation of new techniques.

\paragraph{Applying state-decomposition to policy gradient algorithms} The state-decomposition method was mostly applied in this
project to methods based on deep Q-learning. However, state-decomposition is not limited to DQN and it could for example be
applied to policy gradient methods.