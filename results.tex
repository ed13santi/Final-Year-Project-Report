\chapter{Testing}
Describe your Test Plan -- how the program or system
was verified. Put the actual test results in an Appendix
if they are repetitive but relevant. Detailed test data
may be omitted from the report if not relevant,
however an accurate summary of tests should be
included in the Report itself. Sometimes non-working
designs are described in project reports as though
they work, when in reality they don’t, or only partially
work. Therefore a precise description of what works
and how this has been established is important.
Examiners may try to compile, use, or test
deliverables themselves (even after your report is
submitted), and your report should accurately reflect
the state of the project.
This section is normally useful for software or
hardware deliverables and less relevant in analytical
projects.

\chapter{Results}
This covers an area different from the 'Testing'
chapter, and relates to the understanding and analysis
of the algorithm, program, or hardware behaviour.
Where the deliverable is a product with easily
quantified performance then the previous Chapter
shows how functional correctness is established,
while this Chapter shows qualitatively or
quantitatively how well the product works. The list
below shows typical content, not all of this will be
applicable to all projects.
- An empirical relationship between
parameters and results may be investigated,
typically through the use of appropriate
graphs.
- Where theory predicts aspects of
performance the results can be compared 
TJWC Final Year Projects 4
with expectation and differences noted and (if
possible) explained.
- Semi-empirical models of behaviour may be
developed, justified, and tested.
- The types of experiments/simulations that
were carried out should be described. Why
were certain experiments carried out but not
others? What were the important parameters
in the simulation and how did they affect the
results? If there are very many graphs and
tables associated with this chapter they may
be put in the Appendix, but it is generally
better to keep these close to the text they
illustrate, as this is easier for the reader.

\section{Needing to use one-hot encoding}
\section{Initial results with Taxi2 and original state decomposition algorithm}
\section{Effect of different encoding methods}
\section{Effect of different number of parameters}
\section{Effect of using reduced encoding (taking advantage of state decomposition for smaller input vector)}
\section{Performance gain of DeltaSwitch}
\section{Samples plots}
\section{Effect of wall / no wall}
\section{Effect of decomposing by destination or by position}
\section{Techniques to learn switch and relationship with transfer learning + fair comparison}

\chapter{Evaluation Plan}

%Differently from many other projects, this project does not have as a goal a certain application for which different techniques
%can be used. Thus the eventual success of the project can't be evaluated by the success of the application. In this project, we
%are developing the technique itself, which could then be applied in different contexts. Since the project consists of the
%development of a new algorithm, there is no certainty that the algorithm would achieve good performance. Thus, the project can be
%considered successful even if the state-decomposition algorithm does not show improved performance.

The main objective of this project is to evaluate the performance of the proposed state-decomposition algorithm. To do so, the
algorithm will be tested in different varieties in the TaxiTraps environment described in Section-\ref{sec:TaxiTraps}. The
possible varieties are:

\begin{itemize}
    \item In the original state-decomposition algorithm it is assumed that the NNs of the sub-spaces' agents keep training after
    they are joined together in the second stage of the training as they are part of the bigger newly formed neural network of the
    global agent. Alternatively, we could freeze the weights of the sub-spaces' NNs (trained in the 1st stage) during the second
    stage of training. 
    
    \item Once the NNs are joined in the 2nd stage, the training could be done using all transitions, only those that start and
    end in different sub-space\footnote{To take into account the transitions between different sub-spaces.} or start with only
    transitions between different sub-spaces\footnote{At this stage of training we have many of these transitions that were saved
    but ignored in the first stage of training.} and then as the global agent experiences more transitions use all possible
    transitions.
    
    \item The time at which the agents of the sub-spaces should be joined together to form the global agent is not defined.
    Different strategies can be compared, such as waiting for the sub-agents' action-values to converge or defining different
    minimum rates of change of the average episode rewards going below which would start the second stage of training.
\end{itemize}

\paragraph{}
The main aspect of these algorithms that will be evaluated is the sample efficiency, which is how efficiently the algorithm uses
the samples that it experiences to learn a policy. This affects how quickly the algorithm can learn an optimal policy and it can
be measured by plotting the total reward per episode vs the number of training steps. It is also important to consider the fact
that the learning process is stochastic and the performance varies between trials, hence we normally express the reward per
episode using a confidence bound.

\begin{figure}[H]
%\begin{wrapfigure}{r}{0.5\linewidth}
    \centering
    \includegraphics[width=0.5\linewidth]{figures/exampleOfGraph.PNG}
    \caption{Example of reward per episode vs number of training steps. Note the confidence bands to express the uncertainty in
    the results. Figure taken from \cite{stateActionEmbeddings}}
    \label{fig:my_label}
%\end{wrapfigure}
\end{figure}

\paragraph{}
All these variations of the state-decomposition method will be compared with a standard DQN agent. There are many scenarios where
the system is known to be formed by almost independent sub-systems corresponding to almost independent sub-problems. In such a
scenario, the states could be grouped into separate sub-spaces without even estimating the state-transition matrix by just
leveraging the knowledge of the characteristics of the system. Thus, we shall compare the performance of the state-decomposition
method when the state-transition matrix is given (or sub-spaces can be guessed) at the start and when it has to be estimated from
samples. In the latter case, we can estimate the model of the environment is we also consider the rewards of the samples. Thus we
shall also compare the performance of the state-decomposition algorithm with a method that simply approximates the model of the
environment from samples and then finds the optimal policy using planning algorithms. 

THINGS TO CARE ABOUT:

When I use the deltaswitch, how the weights updated for NN0 when NN1 is active and vice-versa.
https://www.gatsby.ucl.ac.uk/~dayan/papers/cjch.pdf https://arxiv.org/pdf/1901.00137.pdf relationship with drpout gains might be
due to effects of double q-learning rather than decomposition resource allocation / job allocation alibaba datasetF


%%%%%%%%%%%%%%%%%%%%%%%%%%%%%%%%%%%%
